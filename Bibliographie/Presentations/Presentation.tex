\documentclass[11pt]{beamer}
% Theme configuration
\usetheme{Warsaw}
\usepackage{ragged2e}
% Packages
\usepackage[utf8]{inputenc}
\usepackage[T1]{fontenc}
\usepackage{comment}
\usepackage[french]{babel}
\usepackage{graphicx}
\usepackage[table,xcdraw]{xcolor}
\usepackage{hyperref}
\usepackage{pdfpages}
\usepackage{multicol}
\usepackage{array}
\usepackage{booktabs}
\usepackage{colortbl}
\usepackage{bibentry}
\usepackage{fontawesome5} % Pour les icônes
\usepackage[backend=bibtex,style=numeric,sorting=none]{biblatex} % Style numérique avec ordre des citations
\addbibresource{bibliography.bib} % Chemin vers votre fichier .bib
\hypersetup{
    colorlinks=true,
    linkcolor=blue,
    urlcolor=blue,
    citecolor=blue
}
\author[DONGMO FEUDJIO - KENMOGNE]{}
% Obtenir le nombre total de slides
\usepackage{etoolbox}
\makeatletter
\patchcmd{\beamer@enddocumenthook}{\clearpage}{\clearpage%
\immediate\write\@auxout{\string\newlabel{lastframe}{{}{\theframenumber}}}}{}{}
\makeatother
% Pied de page personnalisé
\setbeamertemplate{footline}{
\leavevmode%
\hbox{%
\begin{beamercolorbox}[wd=.24\paperwidth,ht=2.5ex,dp=1ex,left]{author in head/foot}%
\hspace*{1ex}DONGMO FEUDJIO
\end{beamercolorbox}%
\begin{beamercolorbox}[wd=.64\paperwidth,ht=2.5ex,dp=1ex,center]{title in head/foot}%
      Détection de nids de poules
\end{beamercolorbox}%
\begin{beamercolorbox}[wd=.09\paperwidth,ht=2.5ex,dp=1ex,center]{date in head/foot}%
\insertframenumber{} / \inserttotalframenumber\hspace*{1ex}
\end{beamercolorbox}
  }%
\vskip0pt%
}
\logo{\includegraphics[height=6mm]{img/logo.png}}
% Title page configuration
\title{Détection de nids de poules sur le tronçon Dschang-Bafoussam}
\begin{document}
% Page de titre
\begin{frame}[plain]
\begin{figure}
\vspace{-0.5cm}
\centering
\includegraphics[width=0.9\textwidth]{img/page_de_garde.png}
\end{figure}
\vspace{-0.9cm}
\date{}
\titlepage
\vspace{-2.25cm}
\centering
\scriptsize
\textbf{Par :}
\vspace{0.2cm}
\begin{tabular}{p{6cm}p{4cm}}
\toprule
\textbf{Nom et Prénom} & \textbf{Matricule} \\
    DONGMO FEUDJIO Divin Virlence & CM-UDS-21SCI0878 \\
    KENMOGNE Ange Prisca & CM-UDS-21SCI \\
\bottomrule
\end{tabular}\\[1mm]
\textbf{Sous la direction de :}
\vspace{0.2cm}
\begin{tabular}{p{6cm}p{4cm}}
\toprule
\textbf{Nom et Prénom} & \textbf{Statut} \\
    Pr KENGNE TCHENDJI Vianney & Maître de Conférences \\
    YAWENI & Chargé de TD \\
\bottomrule
\end{tabular}\\[1mm]
\normalsize
\today
\end{frame}
\begin{frame}[plain]
\frametitle{Plan de présentation}
\tableofcontents[hideallsubsections]
\end{frame}
\section{Analyse du papier : You only look once: Unified, real-time object detection}
\begin{frame}[allowframebreaks]
\frametitle{Analyse du papier : You Only Look Once (YOLO)}
\framesubtitle{Résumé et Contexte}
\textbf{Papier de Redmon et al. (2016)} : Introduit YOLO pour détection en temps réel.
\begin{itemize}
\item \textbf{Approche} : Régression unique vs. pipelines complexes (R-CNN).
\item \textbf{Points clés} :
\begin{itemize}
\item Vitesse : 45 FPS (base), 155 FPS (Fast YOLO).
\item Précision : Double mAP temps réel, mais + erreurs localisation.
\item Généralisation : Meilleur sur art vs. DPM/R-CNN.
\end{itemize}
\end{itemize}
\end{frame}

\begin{frame}
\begin{figure}[h!]
\centering
\includegraphics[width=1\textwidth]{img/YOLO_System_Detection.png}
\caption{Système de détection de YOLO \cite{redmon2016you}}
\end{figure}
\begin{itemize}
  \item Redimensionner l'image à 448x448
  \item Exécuter un réseau convolutif unique
  \item Seuiller les détections par confiance
\end{itemize}
\end{frame}

\begin{frame}
\frametitle{Analyse du papier : You Only Look Once (YOLO)}
\framesubtitle{Introduction (Section 1)}
\textbf{Problématique} : Système humain rapide et précis vs. méthodes lentes (DPM, R-CNN) qui réutilisent des capteurs pour la détection.
\begin{itemize}
\item \textbf{YOLO} : Une passe pour prédictions objets/positions.
\item \textbf{Avantages} :
\begin{itemize}
\item Vitesse : 45-150 FPS, latence <25 ms.
\item Raisonnement global : $\frac{1}{2}$ faux positifs arrière-plan vs. Fast R-CNN.
\item Généralisation : Domaines inattendus (art).
\end{itemize}
\item \textbf{Limites} : Localisation imprécise petits objets.
\end{itemize}
\end{frame}
\begin{frame}[allowframebreaks]
\frametitle{Analyse du papier : You Only Look Once (YOLO)}
\framesubtitle{Détection Unifiée (Section 2)}
\textbf{Architecture} : Grille $S\times S$ et si le centre d'un objet tombe dans une cellule de grille, cette cellule de grille est responsable de la détection de cet objet.
\begin{itemize}
\item Chaque \textbf{cellule de grille} prédit B boîtes englobantes, des scores de confiance pour ces boîtes et C probabilités de classes conditionnelles.
\item \textbf{Score} : $Pr(Class_i|Object) \times Pr(Object) \times IOU$.
\item Chaque boîte englobante consiste en 5 prédictions : \textbf{x, y, w, h, et la confiance}
\begin{itemize}
\item x,y : les coordonnées du centre de la boîte
\item w,h : la largeur et la hauteur de la boîte
\item la confiance : IOU entre la boîte prédite et toute boîte vérité terrain
\end{itemize}

\begin{figure}[h!]
\centering
\includegraphics[width=0.8\textwidth]{img/YOLO_Model.png}
\caption{Modèle de détection de YOLO \cite{redmon2016you}}
\end{figure}

\begin{itemize}
\item x,y : les coordonnées du centre de la boîte
\item w,h : la largeur et la hauteur de la boîte
\item la confiance : IOU entre la boîte prédite et toute boîte vérité terrain
\end{itemize}

\item \textbf{Réseau} : ce modèle est implémenté comme un réseau neuronal convolutif.
\begin{figure}[h!]
\centering
\includegraphics[width=0.9\textwidth]{img/YOLO_Model_Reseau.png}
\caption{Modèle de détection de YOLO \cite{redmon2016you}}
\end{figure}
\item \textbf{Entraînement} :
\begin{itemize}
\item Pré-entraînement des couches convolutives sur ImageNet (1000 classes).
\item Conversion du modèle pour la détection en ajoutant 4 couches convolutives et 2 couches entièrement connectées.
\item Augmentation de la résolution d'entrée de 224x224 à 448x448 pour des informations visuelles fines.
\item Utilisation d'une fonction de perte somme des carrés pondérée ($\lambda_{coord}=5$, $\lambda_{noobj}=0.5$).
\item Prédiction de la racine carrée de w et h pour mieux gérer les erreurs sur petites boîtes.
\item Assignation d'un prédicteur responsable par objet basé sur l'IOU le plus élevé.
\end{itemize}
\item \textbf{Inférence} :
\begin{itemize}
\item Une seule évaluation du réseau pour prédire 98 boîtes et classes par image.
\item Utilisation de NMS pour éliminer les détections multiples (+2-3\% mAP).
\item Pas de pipeline complexe, rapide pour le streaming vidéo.
\end{itemize}
\item \textbf{Limites} :
\begin{itemize}
\item Contraintes spatiales : seulement 2 boîtes et 1 classe par cellule, limite les objets proches.
\item Difficulté avec petits objets groupés (ex. : troupeaux d'oiseaux).
\item Généralisation limitée à ratios d'aspect ou configurations inhabituelles.
\item Caractéristiques grossières dues au downsampling.
\item Fonction de perte traite erreurs identiquement pour petites/grandes boîtes (source principale d'erreurs de localisation).
\end{itemize}
\end{itemize}
\end{frame}
\begin{frame}[allowframebreaks]
\frametitle{Analyse du papier : You Only Look Once (YOLO)}
\framesubtitle{Comparaison systèmes (Section 3)}
\textbf{YOLO vs. concurrents :}
\begin{itemize}
\item \textbf{YOLO vs. DPM} : Architecture unifiée vs. approche de fenêtre glissante ; remplace extraction statique, classification et prédiction par un seul CNN optimisé pour la détection.
\item \textbf{YOLO vs. R-CNN} : Contraintes spatiales sur propositions (98 vs. 2000) ; optimisation conjointe vs. pipeline disjoint lent (>40s/image).
\item \textbf{YOLO vs. Fast/Faster R-CNN} : Plus rapide (45 FPS vs. 7 FPS) mais moins précis ; ne repose pas sur Selective Search.
\item \textbf{YOLO vs. MultiBox} : Système complet vs. seulement prédiction de régions ; détection générale vs. pièce de pipeline.
\item \textbf{YOLO vs. OverFeat} : Raisonnement global vs. local ; optimise pour détection vs. localisation.
\item \textbf{YOLO vs. MultiGrasp} : Détection multiple objets/classes vs. une seule région saisissable.
\item \textbf{Tableau comparatif :}
\end{itemize}

\begin{tabular}{lcc}
\toprule
Méthode & mAP & FPS \\
\midrule
Fast YOLO & 52.7 & 155 \\
YOLO & 63.4 & 45 \\
Fast R-CNN & 70.0 & 0.5 \\
\bottomrule
\end{tabular}
\end{frame}
\begin{frame}[allowframebreaks]
\frametitle{Analyse du papier : You Only Look Once (YOLO)}
\framesubtitle{Expériences (Section 4)}
\textbf{Résultats VOC 2007 :}
\begin{itemize}
\item \textbf{4.1 Temps réel} : YOLO 63.4 mAP @45 FPS (2x DPM).
\item \textbf{4.2 Erreurs} :
\begin{itemize}
\item YOLO : 19\% loc., 4.75\% bg.
\item Fast R-CNN : 8.6\% loc., 13.6\% bg.
\end{itemize}
\item \textbf{4.3 Combinaison} : +3.2\% mAP (75.0\%).
\item \textbf{4.4 VOC 2012} : 57.9 mAP (faible petits objets).
\item \textbf{4.5 Généralisation} : Art Picasso 59\% AP (meilleur R-CNN).
\end{itemize}
\end{frame}

\begin{frame}[allowframebreaks]
\frametitle{Analyse du papier : You Only Look Once (YOLO)}
\framesubtitle{Sections 5-6 \& Applicabilité}
\textbf{Section 5 (Détection en temps réel dans le monde réel)} : 
\begin{itemize}
\item YOLO est connecté à une webcam pour démontrer ses performances en temps réel, y compris la récupération des images et l'affichage des détections.
\item Le système est interactif : traite les images individuellement mais fonctionne comme un suivi en détectant les objets en mouvement.
\item Démo et code source disponibles sur le site du projet (open-source).
\end{itemize}

\textbf{Section 6 (Conclusion)} : 
\begin{itemize}
\item YOLO est un modèle unifié, simple à construire, entraîné directement sur images complètes.
\item Optimisé pour performances de détection ; rapide, précis et généralisable.
\item Pousse l'état de l'art en détection temps réel ; idéal pour applications nécessitant robustesse.
\end{itemize}
\end{frame}

\begin{frame}[allowframebreaks]
\frametitle{Références}
\begin{scriptsize}
\printbibliography[title={}]
\end{scriptsize}
\end{frame}
\end{document}