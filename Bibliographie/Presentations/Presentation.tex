\documentclass[11pt]{beamer}
% Theme configuration
\usetheme{Warsaw}
\usepackage{ragged2e}
% Packages
\usepackage{pdftexcmds}
\usepackage[utf8]{inputenc}
\usepackage[T1]{fontenc}
\usepackage{comment}
\usepackage[french]{babel}
\usepackage{graphicx}
\usepackage[table,xcdraw]{xcolor}
\usepackage{hyperref}
\usepackage{pdfpages}
\usepackage{multicol}
\usepackage{array}
\usepackage{booktabs}
\usepackage{colortbl}
\usepackage{bibentry}
\usepackage{fontawesome5} % Pour les icônes
\usepackage[backend=bibtex,style=numeric,sorting=none]{biblatex} % Style numérique avec ordre des citations
\addbibresource{bibliography.bib} % Chemin vers votre fichier .bib
\hypersetup{
    colorlinks=true,
    linkcolor=blue,
    urlcolor=blue,
    citecolor=blue
}
\author[DONGMO FEUDJIO]{}
% Obtenir le nombre total de slides
\usepackage{etoolbox}
\makeatletter
\patchcmd{\beamer@enddocumenthook}{\clearpage}{\clearpage%
\immediate\write\@auxout{\string\newlabel{lastframe}{{}{\theframenumber}}}}{}{}
\makeatother
% Pied de page personnalisé
\setbeamertemplate{footline}{
\leavevmode%
\hbox{%
\begin{beamercolorbox}[wd=.24\paperwidth,ht=2.5ex,dp=1ex,left]{author in head/foot}%
\hspace*{1ex}DONGMO FEUDJIO
\end{beamercolorbox}%
\begin{beamercolorbox}[wd=.64\paperwidth,ht=2.5ex,dp=1ex,center]{title in head/foot}%
      Détection de nids de poule
\end{beamercolorbox}%
\begin{beamercolorbox}[wd=.09\paperwidth,ht=2.5ex,dp=1ex,center]{date in head/foot}%
\insertframenumber{} / \inserttotalframenumber\hspace*{1ex}
\end{beamercolorbox}
  }%
\vskip0pt%
}
\logo{\includegraphics[height=6mm]{img/logo.png}}

% Title page configuration
\title{Détection de nids de poule sur le tronçon Dschang-Bafoussam}
\begin{document}
% Page de titre
\begin{frame}[plain]
\begin{figure}
\vspace{-0.5cm}
\centering
\includegraphics[width=0.9\textwidth]{img/page_de_garde.png}
\end{figure}
\vspace{-0.9cm}
\date{}
\titlepage
\vspace{-2.25cm}
\centering
\scriptsize
\textbf{Par :}
\vspace{0.2cm}
\begin{tabular}{p{6cm}p{4cm}}
\toprule
\textbf{Nom et Prénom} & \textbf{Matricule} \\
    DONGMO FEUDJIO Divin Virlence & CM-UDS-21SCI0878 \\
    KENMOGNE Ange Prisca & CM-UDS-21SCI \\
\bottomrule
\end{tabular}\\[1mm]
\textbf{Sous la direction de :}
\vspace{0.2cm}
\begin{tabular}{p{6cm}p{4cm}}
\toprule
\textbf{Nom et Prénom} & \textbf{Statut} \\
    Pr KENGNE TCHENDJI Vianney & Maître de Conférences \\
    YAWENI & Chargé de TD \\
\bottomrule
\end{tabular}\\[1mm]
\normalsize
\today
\end{frame}

\begin{frame}[plain]
    \frametitle{Plan de présentation}
    \tableofcontents[hideallsubsections]
\end{frame}

\section{Analyse du papier : You only look once: Unified, real-time object detection}

\begin{frame}[allowframebreaks]
    \frametitle{Analyse du papier : You Only Look Once (YOLO)}
    \framesubtitle{Résumé et Contexte}
    \textbf{Papier de Redmon et al. (2016)} : Introduit YOLO pour détection en temps réel.
    \begin{itemize}
        \item \textbf{Approche} : Régression unique vs. pipelines complexes (R-CNN).
        \item \textbf{Points clés} :
        \begin{itemize}
            \item Vitesse : 45 FPS (base), 155 FPS (Fast YOLO).
            \item Précision : Double mAP temps réel, mais + erreurs localisation.
        \end{itemize}
    \end{itemize}
\end{frame}

\begin{frame}
    \frametitle{Analyse du papier : You Only Look Once (YOLO)}
    \framesubtitle{Système de détection YOLO}

    \begin{figure}[h!]
        \centering
        \includegraphics[width=1\textwidth]{img/YOLO_System_Detection.png}
        \caption{Système de détection de YOLO \cite{redmon2016you}}
    \end{figure}

    \begin{itemize}
        \item Division de l’image en grille
        \item Prédiction simultanée par chaque cellule de chaque grille
        \item Post-traitement avec suppression non-maximale (NMS)
    \end{itemize}
\end{frame}


\begin{frame}
    \frametitle{Analyse du papier : You Only Look Once (YOLO)}
    \framesubtitle{Division de l’image en grille (Grid Division)}
    L'image est divisée en grille $S\times S$ et si le centre d'un objet tombe dans une cellule de grille, cette cellule de grille est responsable de la détection de cet objet.

    \begin{figure}[h!]
        \centering
        \includegraphics[width=0.8\textwidth]{img/YOLO_Model.png}
        \caption{Modèle de détection de YOLO \cite{redmon2016you}}
    \end{figure}
\end{frame}


\begin{frame}
    \frametitle{Analyse du papier : You Only Look Once (YOLO)}
    \framesubtitle{Prédiction simultanée par chaque cellule de chaque grille}
    \begin{itemize}
        \item Chaque \textbf{cellule de grille} prédit B boîtes englobantes, des scores de confiance pour ces boîtes et C probabilités de classes conditionnelles.
        \item \textbf{Score} : $Pr(Class_i|Object) \times Pr(Object) \times IOU$.
        \item Chaque boîte englobante consiste en 5 prédictions : \textbf{x, y, w, h, et la confiance}
        \begin{itemize}
            \item x,y : les coordonnées du centre de la boîte
            \item w,h : la largeur et la hauteur de la boîte
            \item la confiance : IOU entre la boîte prédite et toute boîte vérité terrain
        \end{itemize}
    \end{itemize}
\end{frame}

\begin{frame}
    \frametitle{Analyse du papier : You Only Look Once (YOLO)}
    \framesubtitle{Non-Maximum Suppression (NMS)}
    Après prédiction, plusieurs boîtes peuvent détecter le même objet.

    \textbf{Étapes du NMS} :
    \begin{itemize}
        \item Trier les boîtes par ordre décroissant de \textbf{score de confiance}.
        \item Sélectionner la boîte avec le score le plus élevé.
        \item Supprimer toutes les boîtes avec \textbf{IoU > seuil} (généralement 0.5) avec celle-ci.
        \item Répéter jusqu’à épuisement.
    \end{itemize}

    \textbf{Résultat :} une seule détection finale par objet, sans doublons.
\end{frame}

\begin{frame}[allowframebreaks]
    \frametitle{Analyse du papier : You Only Look Once (YOLO)}
    \framesubtitle{Architecture du réseau}
    Ce modèle est implémenté comme un réseau neuronal convolutif.

    \begin{figure}[h!]
        \centering
        \includegraphics[width=0.9\textwidth]{img/YOLO_Model_Reseau.png}
        \caption{Modèle de détection de YOLO \cite{redmon2016you}}
    \end{figure}

    \begin{itemize}
        \item \textbf{Entraînement} :
        \begin{itemize}
            \item Pré-entraînement des couches convolutives.
            \item Conversion du modèle pour la détection en ajoutant 4 couches convolutives et 2 couches entièrement connectées.
            \item Augmentation de la résolution d'entrée de 224x224 à 448x448 pour des informations visuelles fines.
            \item Utilisation d'une fonction de perte somme des carrés pondérée ($\lambda_{coord}=5$, $\lambda_{noobj}=0.5$).
            \item Prédiction de la racine carrée de w et h pour mieux gérer les erreurs sur petites boîtes.
            \item Assignation d'un prédicteur responsable par objet basé sur l'IOU le plus élevé.
        \end{itemize}
        \item \textbf{Avantages} :
        \begin{itemize}
            \item Une seule évaluation du réseau pour prédire 98 boîtes et classes par image.
            \item Utilisation de NMS pour éliminer les détections multiples.
            \item Pas de pipeline complexe, rapide pour le streaming vidéo.
        \end{itemize}
        \item \textbf{Limites} :
        \begin{itemize}
            \item Contraintes spatiales : seulement 2 boîtes et 1 classe par cellule, limite les objets proches.
            \item Difficulté avec petits objets groupés (ex. : troupeaux d'oiseaux).
        \end{itemize}
    \end{itemize}
\end{frame}


\begin{frame}[allowframebreaks]
    \frametitle{Analyse du papier : You Only Look Once (YOLO)}
    \framesubtitle{Fonction de perte}

    YOLO optimise une \textbf{fonction de perte multi-parties} basée sur l'erreur quadratique (sum-squared error). Cette fonction est composée de cinq termes distincts, chacun correspondant à un objectif d'apprentissage différent.

    La fonction complète est la suivante :

    \begin{figure}[h!]
        \centering
        \includegraphics[width=0.9\textwidth]{img/fonction_perte.png}
        \caption{Fonction de perte de YOLO \cite{redmon2016you}}
    \end{figure}
\end{frame}


\begin{frame}[allowframebreaks]
    \frametitle{Analyse du papier : You Only Look Once (YOLO)}
    \framesubtitle{Fonction de perte}

    \textbf{Explication détaillée de chaque terme}

    \begin{itemize}
        \item \textbf{Erreur de localisation sur les coordonnées du centre (x, y) = premier terme} \\
        \begin{itemize}
            \item Calcule l'erreur quadratique (MSE) entre le centre prédit ($\hat{x}_i, \hat{y}_i$) et le centre réel ($x_i, y_i$).
            \item $x$ et $y$ sont relatifs à la cellule de la grille (normalisés entre 0 et 1).
        \end{itemize}

        \item \textbf{Erreur de localisation sur la taille de la boîte (w, h) (deuxième terme)} \\
        \begin{itemize}
            \item Même principe, mais pour la largeur ($w$) et la hauteur ($h$).
            \item On applique la \textbf{racine carrée} avant le calcul de l'erreur.
            \item \textbf{Raison de la racine carrée} : une erreur de 10 pixels sur une petite boîte est beaucoup plus grave que sur une grande boîte. La racine carrée rend l'erreur proportionnellement plus sévère pour les petites tailles.
        \end{itemize}

        \item \textbf{Erreur de confiance pour les boîtes contenant un objet (troisième terme)} \\
        \begin{itemize}
            \item Erreur quadratique sur la \textbf{confidence} prédite ($\hat{C}_i$) par rapport à la vraie confidence ($C_i = \text{IOU}_{\text{pred}}^{\text{truth}}$).
            \item Objectif : apprendre que, quand il y a un objet, la confidence doit être élevée (proche de 1 ou de l'IoU réel).
        \end{itemize}

        \item \textbf{Erreur de confiance pour les boîtes sans objet(quatrième terme)} \\
        \begin{itemize}
            \item Même erreur de confidence, mais pour les cellules/boîtes \textbf{sans objet} ($C_i = 0$).
            \item Objectif : pénaliser les faux positifs (prédire un objet là où il n'y en a pas).
        \end{itemize}

        \item \textbf{Erreur de classification (probabilités de classes) (cinquième terme)} \\
        \begin{itemize}
            \item Erreur quadratique sur les probabilités conditionnelles $\text{Pr}(\text{Class}_c \mid \text{Objet})$.
            \item On compare la probabilité prédite ($\hat{p}_i(c)$) à la vraie (1 pour la classe correcte, 0 sinon).
            \item Utilise MSE pour simplicité (les versions modernes passent à la cross-entropy).
        \end{itemize}
    \end{itemize}
\end{frame}

\begin{frame}[allowframebreaks]
    \frametitle{Analyse du papier : You Only Look Once (YOLO)}
    \framesubtitle{Fonction de perte}

    \textbf{Résumé des objectifs de la fonction de perte}

    \begin{itemize}
        \item \textbf{Localisation} (termes 1 et 2) → apprendre à bien placer et dimensionner les boîtes (fortement pondérée).
        \item \textbf{Confiance avec objet} (terme 3) → apprendre à être confiant quand il y a un objet.
        \item \textbf{Confiance sans objet} (terme 4) → apprendre à ne pas être confiant quand il n'y a rien (pénalité réduite).
        \item \textbf{Classification} (terme 5) → apprendre à identifier correctement la classe de l’objet (uniquement quand objet présent).
    \end{itemize}

Cette conception astucieuse  est la raison pour laquelle YOLO parvient à équilibrer vitesse et précision malgré sa simplicité.

\end{frame}


\begin{frame}
    \frametitle{Analyse du papier : You Only Look Once (YOLO)}
    \framesubtitle{En conclusion}
    \textbf{En conclusion} : 
    \begin{itemize}
        \item YOLO est un modèle unifié, simple à construire, entraîné directement sur images complètes.
        \item Optimisé pour performances de détection ; rapide, précis et généralisable.
        \item Pousse l'état de l'art en détection temps réel ; idéal pour applications nécessitant robustesse.
    \end{itemize}
\end{frame}



\begin{frame}[plain]
    \frametitle{Plan de présentation}
    \tableofcontents[hideallsubsections]
\end{frame}


\section{Projet de détection de nids de poule sur le tronçon de route Dschang-Bafoussam}

\begin{frame}
    \frametitle{Objectifs du projet}
    \framesubtitle{Contexte et problématique}

    \begin{block}{Problématique principale}
        Les nids-de-poule constituent un danger majeur sur les routes camerounaises, notamment sur l'axe \textbf{Dschang–Bafoussam}.
    \end{block}

    \bigskip

    \textbf{Objectifs du projet :}
    \begin{itemize}
        \item Détecter automatiquement les nids-de-poule en temps réel à partir de vidéos de dashcam
        \item Avertir le conducteur (alerte visuelle / sonore)
        \item Cartographier les zones à risque pour les autorités
        \item Quantifier les dommages routiers
    \end{itemize}
    \end{frame}
\end{frame}

\begin{frame}
    \frametitle{Choix du dataset public}
    
    \begin{itemize}
        \item Dataset : \mymark{Potholes-Detection-YOLOv8}
        \item Auteur : Angga Dwi Sunarto
        \item Plateforme : Kaggle
        \item Lien : \url{https://www.kaggle.com/datasets/anggadwisunarto/potholes-detection-yolov8}
        \item Format natif : \important{YOLOv8}
        \item Nombre d'images : 
        \begin{itemize}
            \item Train : 1581
            \item Valid : 396
        \end{itemize}
    \end{itemize}
\end{frame}

\begin{frame}{Import et traitement via Roboflow}
    \begin{itemize}
        \item Création d'un compte sur \href{https://roboflow.com}{Roboflow}
        \item Projet : \mymark{Pothole Cameroon}
        \item Upload du ZIP Kaggle complet
        \item Redimensionnement : \important{512 × 512}
        \item Nouvelle version du dataset : \good{images\_ok}
        \item Split final : \textbf{70\% – 15\% – 15\%}
        \item Export au format \textbf{YOLOv8}
    \end{itemize}

    \bigskip

    Lien du dataset final :  
    \url{https://app.roboflow.com/pothole-detection-wipyl/pothole-detection-dschang-bafous/1}
\end{frame}

\begin{frame}{Environnement d'entraînement}
    \begin{block}{Configuration utilisée}
    \begin{itemize}
        \item Plateforme : Google Colab (GPU)
        \item GPU : Tesla T4
        \item Bibliothèque : \important{Ultralytics YOLOv8}
        \item Modèle de base : \textbf{yolov8n.pt} (nano – très léger)
        \item Nombre d'epochs : \good{100}
    \end{itemize}
    \end{block}

    \bigskip

    Notebook complet :  
    \url{https://github.com/VirlenceDongmo/Pothole_Detection/blob/master/Code/pothole_notebook/Pothole_detection.ipynb}
\end{frame}

\begin{frame}{Interprétation des métriques clés}
    \begin{columns}[T]
        \column{0.52\textwidth}
        \begin{itemize}
            \item \good{mAP@0.5 = 75.8\%} \quad → excellent
            \item Precision = \good{76.8\%}
            \item Recall = 68.3\%
            \item mAP@0.5:0.95 = 48.8\%
        \end{itemize}

        \column{0.48\textwidth}
        \begin{block}{Interprétation}
        \begin{itemize}
            \item Très bon compromis précision / rapidité
            \item Peu de faux positifs → alertes fiables
            \item Recall perfectible sur petits / mal éclairés potholes
        \end{itemize}
        \end{block}
    \end{columns}
\end{frame}

\begin{frame}{Application web de démonstration}
    \begin{itemize}
        \item Framework : \textbf{Flask}
        \item Fonctionnalités :
        \begin{itemize}
            \item Upload vidéo
            \item Détection en temps réel
            \item Visualisation des résultats
        \end{itemize}
        \item Modèle chargé : \good{best.pt}
    \end{itemize}
\end{frame}

\begin{frame}{Application web de démonstration}
    \begin{figure}
        \centering
        \includegraphics[width=0.95\textwidth]{img/page_accueil.png}
        \caption{Page d'accueil de l'application}
    \end{figure}
\end{frame}

\begin{frame}{Synthèse et résultats obtenus}
\begin{block}{Points forts}
\begin{itemize}
    \item mAP@0.5 \important{75.8\%} sur un dataset mixte
    \item Déploiement fonctionnel (Flask)
    \item Code et données publics sur GitHub
\end{itemize}
\end{block}

\bigskip

Lien du projet complet :  
\href{https://github.com/VirlenceDongmo/Pothole_Detection}{\faGithub \quad VirlenceDongmo / Pothole\_Detection}
\end{frame}

\begin{frame}{Perspectives d'amélioration}
\begin{itemize}
    \item Collecte et annotation d'images locales 
    \item Augmentation des données (pluie, nuit, poussière, ombre)
    \item Intégration en temps réel dans un véhicule 
    \item Développement d'une application mobile
    \item Création d'une carte interactive des nids-de-poule (base de données géolocalisée)
\end{itemize}
\end{frame}






\begin{frame}
    \frametitle{Références}
    \printbibliography
\end{frame}

\end{document}